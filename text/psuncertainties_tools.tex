\section{Tools}
\label{sec:psunc:tools}

\section{\Deductor}
\label{sec:psunc:tools:decuctor}

Deductor is a dipole shower with virtuality based ordering. It uses a color
aproximation that goes beyond leading color (LC), namely the LC+
approximation. This includes some -- but certainly not all -- corrections to
order $1/N_c^4$. However, I tried the 500 GeV case with just LC and found that
there is no substantial difference for the observables considered between LC
and LC+. Rather, differences with other parton shower algorithms presumably
come from a different ordering variable, different splitting functions, and a
different recoil scheme.

\section{\Herwig}
\label{sec:psunc:tools:herwig}

The \Herwig~7 event generator \cite{Bellm:2015jjp} is currently offering two
shower modules: the traditional, angular ordered parton shower as set out in
\cite{Gieseke:2003rz}, as well as a dipole-type shower based on the work
presented in \cite{Platzer:2009jq,Platzer:2011bc}.
Though very different in their nature, both showers guarantee coherent
evolution and reach a similar level of description of data when interfaced to
the cluster hadronization model. The CMW prescription \cite{Catani:1990rr} is
not implemented directly, but the relevant change in $\Lambda$ is considered
to be absorbed into using a tuned value of $\alpha_S(M_Z)$ throughout the
simulation.

As resummation properties of the parton showers strictly beyond leading
logarithmic level are hard to claim (with evidence provided only in limiting
phase space regions \cite{Catani:1990rr}) we do not include terms which modify
the splitting kernels in the presence of variations of the argument of the
strong coupling.

For the parton level comparison at hand, which will have to be related to the
cross-feed with non-perturbative models, we chose to use shower settings which
put the \Herwig~7 showers onto a similar-as-possible level using the same
two-loop running of $\alpha_s$, with $\alpha_s(M_Z)=0.118$, a similar cutoff
prescription and cutoff value ($p_{\perp,\text{min}}=1\ {\rm GeV}$) and not
intrinisic $p_\perp$ generated. The difference to the tuned settings may serve
as a first indicator of the impact of non-perturbative corrections.


\section{\Pythia}
\label{sec:psunc:tools:pythia}

\section{\Sherpa}
\label{sec:psunc:tools:sherpa}

The \Sherpa Monte Carlo event generator \cite{Gleisberg:2008ta} in its latest
release, \Sherpa-2.2.0, comprises two parton shower algorithms: \CSS
\cite{Schumann:2007mg} and \Dire \cite{Hoche:2015sya}. While the \CSS bases on
Catani-Seymour \cite{Catani:1996vz,Catani:2002hc} dipole splitting functions,
\Dire combines the standard treatment of collinear configurations in parton
showers with the resummation of soft logarithms in color dipole cascades. A
third parton shower, \Ants \cite{ants}, is under development. \Ants bases on
dipole splitting functions in the spirit of
\cite{Winter:2007ye,Lonnblad:1992tz}.

In general the splitting functions of all three algorithms take the 
form 
\begin{equation}
  \begin{split}\label{eq:psunc:tools:sherpa:sf}
    \mathrm{D}_{ijk}(t,z,\phi)
    \,=\;& \frac{\alpha_s(k_\text{tune}b\,t)}{2\pi t}\;\mathrm{P}_{ijk}(z)
  \end{split}
\end{equation}
where $b=1$ for \Dire. For \CSS and \Ants, $b=k_\text{CMW}$, where
\begin{equation}
  \begin{split}\label{eq:psunc:tools:sherpa:kcmw}
    k_\text{CMW}
    \,=\;&\exp\left[-\frac{67-3\pi^2-\tfrac{1}{3}\,n_f(t)}{33-2n_f(t)}\right]
  \end{split}
\end{equation}
is the Catani-Marchesini-Webber scale factor \cite{Catani:1990rr}
to incorporate dominating higher-logarithmic contributions and $n_f(t)$ 
is the number of active flavours at scale $t$. The correction originating from
$k_\text{CMW}$ is included in \Dire by multiplying the soft enhanced term of the
splitting functions with $1+\alpha_s/(2\pi)\beta_0k_\text{CMW}$. The $k_\text{tune}$ 
are manually inserted scale factors to accommodate one more degree-of-freedom 
in a tuning context. However, in all current versions of the programs 
$k_\text{tune}$ is fixed to 1 for final state splittings and $\tfrac{1}{2}$ 
for initial state splittings in the \CSS, while \Ants and \Dire employ 
$k_\text{tune}=1$ throughout. The $\mathrm{P}_{ijk}(z)$ are the shower 
dependent splitting functions. 
Since PDF uncertainties are not addressed here, cf.\ \ref{}, ratios of 
PDFs are absorbed into the $\mathrm{P}_{ijk}(z)$.

For the present study the showers run in their default setting.


When now varying the argument of the strong coupling constant, i.e.\ 
replacing $t\to ct$ the higher-logarithmic structure induced by the running 
of the coupling constant in the presence of the CMW scale factor needs to be 
preserved in order not to upset the resummation quality of the parton 
shower. Thus, the following counter term is introduced
\begin{equation}
  \begin{split}\label{eq:psunc:tools:sherpa:asct}
    \alpha_s(k_\text{tune}k_\text{CMW}\,t)
    \,\to\;& \alpha_s(k_\text{tune}k_\text{CMW}\, c\cdot t)\cdot f(c,t)
  \end{split}
\end{equation}
where two forms for the counter term $f(c,t)$ are implemented,
\begin{equation}
  \begin{split}\label{eq:psunc:tools:sherpa:asctfac}
    f(c,t)
    \,=\;&\left\{\begin{array}{ll}
                  1-\sum_{i=0}^{n_\text{th}+1}\frac{\alpha_s}{2\pi}\,\beta_0(n_f(t))\,\log\frac{t_i}{t_{i-1}} & \text{additive threshold treatment}\\
                  \prod_{i=0}^{n_\text{th}+1}\left(1-\frac{\alpha_s}{2\pi}\,\beta_0(n_f(t))\,\log\frac{t_i}{t_{i-1}}\right) & \text{multiplicative threshold treatment.}
                 \end{array}\right.
  \end{split}
\end{equation}
Therein, the sum and product run over the number $n_\text{th}$ of parton mass
thresholds in the interval $[t,c\cdot t]$ with $t_0=t$,
$t_{n_\text{th}+1}=c\cdot t$ and $t_i$ are the encompassed parton mass
thresholds. If $c<1$, then the ordering is reversed recovering the correct
sign. $\beta(n_f)$ is the QCD beta function. Obviously, both forms coincide if
the interval $[t,c\cdot t]$ contains zero or one parton mass thresholds. For
this study the additive threshold treatment was used. We vary the
renormalization scale by a factor two, i.e.\ $c=4$ and $c=1/4$
in~\ref{eq:psunc:tools:sherpa:asct}.

